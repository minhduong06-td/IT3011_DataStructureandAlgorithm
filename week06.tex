\documentclass[aspectratio=169]{beamer}
\usetheme[theme=blue,logo=husty70color]{HUST} 

\usepackage[T5]{fontenc}
\usepackage[utf8]{inputenc}
% \usepackage[utf8]{vietnam} % Uncomment nếu cần thiết tuỳ hệ thống
\usepackage{amsmath}
\usepackage{amsfonts}
\usepackage{amssymb}
\usepackage{graphicx}
\usepackage{adjustbox}
\usepackage{minted}

\setminted{
  breaklines=false,
  autogobble=false,
  obeytabs=true,
  tabsize=2,
  linenos=true,
  showspaces=false,
  space=~,
  baselinestretch=1,
  fontsize=\normalsize
}

\newcommand{\placecontent}[4]{%
  \tikz[remember picture,overlay]
    \node[anchor=north west]
      at ([xshift=#1,yshift=-#2]current page.north west)
      {\parbox{#3}{#4}};
}

\graphicspath{{./week06_resources/}}

\title{\huge CẤU TRÚC DỮ LIỆU VÀ GIẢI THUẬT}
\author{SoICT - HUST}
\date{}

\begin{document}

% 2 slides đầu tiên:
\HUSTInsertBrandSlide
\HUSTInsertThemeSlide

% Slide tiêu đề
{\HUSTUseBackground{onelove.pdf}
\begin{frame}
  \ifdefstring{\insertaspectratio}{169}{
    \HUSTCornerImage[0.14]{assets/logo/soict_vi_h.pdf}
    \placecontent{0.5cm}{0.33\paperheight}{0.85\paperwidth}{
        \color{\HUSTFrameTitleTextColor}\bfseries\fontsize{22pt}{30pt}\selectfont
        \inserttitle
    }
    \placecontent{0.5cm}{0.50\paperheight}{0.8\paperwidth}{
        \color{\HUSTFrameTitleTextColor}\fontsize{14pt}{14pt}\selectfont
        %Bài học
        \textbf{\large Bài 6: Thuật toán Quy hoạch động (Dynamic Programming)}\\
    }
  }{}
\end{frame}
}

% Outline
\AtBeginSection[]
{
    \begin{frame}<beamer>
        \frametitle{Nội dung}
        \tableofcontents[currentsection]
    \end{frame}
}

%Nội dung chính trong slides
\section{Giới thiệu về Quy hoạch động}
\begin{frame}{Giới thiệu}
  \begin{columns}[c] 
    % Cột chứa text (chiếm 65% chiều rộng)
    \begin{column}{0.65\textwidth}
      \begin{itemize}
        \item \textbf{Lịch sử:} Thuật toán quy hoạch động (Dynamic Programming - DP) được phát minh bởi Bellman trong thế chiến thứ 2. Tên ban đầu là \textit{multi-stage decision process} (quá trình ra quyết định qua nhiều giai đoạn).
        \item \textbf{Định nghĩa:} Là kỹ thuật mạnh để giải bài toán tối ưu bằng cách chia thành các bài toán nhỏ hơn và giải các bài toán con \textbf{một lần duy nhất}.
        \item \textbf{So sánh:} Có nhiều điểm giống với:
        \begin{itemize}
            \item Thuật toán Quay lui (Backtracking).
            \item Thuật toán Chia để trị (Divide and Conquer).
        \end{itemize}
      \end{itemize}
    \end{column}
    
    % Cột chứa ảnh (chiếm 33% chiều rộng)
    \begin{column}{0.33\textwidth}
      \centering
      
      \includegraphics[width=\textwidth]{Bellman.jpg} 
    \end{column}
  \end{columns}
\end{frame}
%Viết nó ko đc hay
\begin{frame}{Đặc điểm}
  \begin{columns}[c] 
    % Cột chứa text (chiếm 65% chiều rộng)
    \begin{column}{0.65\textwidth}
  \begin{itemize}
    \item \textbf{CHIA} bài toán xuất phát thành các bài toán con không nhất thiết
độc lập với nhau
    \item \textbf{GIẢI} các bài toán con từ nhỏ đến lớn, lời giải được lưu trữ lại vào
bộ nhớ (để đảm bảo mỗi bài toán chỉ giải đúng 1 lần)
    \begin{itemize}
        \item Bài toán con nhỏ nhất phải được giải một cách
trực tiếp, đơn giản
    \end{itemize}
    \item \textbf{KẾT HỢP} lời giải của bài toán lớn hơn từ lời giải đã có của các bài
toán con nhỏ hơn (cần sử dụng cộng thức truy hồi)
  \end{itemize}
  \textbf{Lưu ý:} Số lượng bài toán con cần được bị chặn bởi một hàm đa thức của kích thước đầu vào.
      \end{column}
    
    % Cột chứa ảnh (chiếm 33% chiều rộng)
    \begin{column}{0.33\textwidth}
      \centering
      
      \includegraphics[width=\textwidth]{DP_1.png} 
    \end{column}
  \end{columns}
\end{frame}
\begin{frame}{Đặc điểm}
    \begin{itemize}
    \item So với thuật toán chia để trị, thuật toán quy hoạch động cũng có 3 bước là Chia, Giải bài toán con
và Kết hợp. Tuy nhiên, trong chia để trị, các bài toán con là độc lập; còn trong quy hoạch động, các
bài toán con gối nhau hay chồng chéo lên nhau.
\vspace{0.5em}
    \item Khó khăn lớn nhất trong đề xuất thuật toán quy hoạch động là Công thức truy hồi (tên khác là Công thức quy hoạch động, Công thức đệ quy)
    \vspace{0.5em}
    \item Có 2 cách tiếp cận: Top-Down và Bottom-Up, trong đó Top-Down tự nhiên và dễ hiểu, dễ cài đặt
toán con nhỏ hơn (cần sử dụng cộng thức truy hồi)
\vspace{0.5em}
    \item Thiết kế bộ nhớ ảnh hưởng lớn tới tốc độ của thuật toán
    \vspace{0.5em}
    \item Bộ nhớ còn dung để truy vết, tìm ra tường minh lời giải tối ưu
  \end{itemize}
  \end{frame}
\section{Sơ đồ thuật toán}


\begin{frame}[fragile]{Mã giả (Pseudo-code)}
Minh họa kỹ thuật \textit{Memoization} (Top-Down):
\begin{minted}[fontsize=\scriptsize, frame=single, framesep=1mm]{cpp}
map<problem, value> Memory; // Bộ nhớ lưu lời giải

value DP(problem P) {
    // 1. Kiểm tra bước cơ sở
    if (is_base_case(P)) 
        return base_case_value(P);
    
    // 2. Kiểm tra bộ nhớ (đã giải chưa?)
    if (Memory.find(P) != Memory.end())
        return Memory[P];

    // 3. Giải bài toán con và kết hợp
    value result = some_value;
    for (problem Q : subproblems(P)) {
        result = Combine(result, DP(Q));
    }

    // 4. Lưu kết quả vào bộ nhớ
    Memory[P] = result;
    return result;
}
\end{minted}
\end{frame}
\begin{frame}{Sơ đồ thuật toán chung}
\centering
      
      \includegraphics[width=\textwidth]{DP_2.png} 
\end{frame}
\section{Ví dụ minh họa}

\subsection{Đoạn con có tổng lớn nhất}
\begin{frame}{Ví dụ 1: Đoạn con có tổng lớn nhất}
  \textbf{Bài toán:} Cho dãy $n$ số nguyên $(a_1, a_2, ..., a_n)$. Hãy tìm đoạn con gồm các phần tử liên tiếp sao cho tổng các phần tử là cực đại.
  
  \vspace{0.5em}
  \textbf{Ví dụ:} Cho dãy 7 số nguyên sau:
  \begin{figure}[H] 
    \centering
    \includegraphics[width=0.8\linewidth]{Ex_1.png}
\end{figure} 
  \begin{itemize}
      \item \textit{Lời giải tối ưu (dãy con có tổng cực đại bằng 8):} $7, -3, 0, -1, 5$ 
  \end{itemize}
  \textbf{Nhận xét:}
  \begin{itemize}
      \item Thuật toán chia để trị có độ phức tạp O (nlog(n)) , chúng ta có thể làm tốt hơn với thuật toán quy
hoạch động hay không?
  \end{itemize}
\end{frame}
\begin{frame}{Ví dụ 1: Đoạn con có tổng lớn nhất}
  \textbf{Bài toán:} Cho dãy $n$ số nguyên $(a_1, a_2, ..., a_n)$. Hãy tìm đoạn con gồm các phần tử liên tiếp sao cho tổng các phần tử là cực đại.
  
  \vspace{0.5em}
  \begin{itemize}
    \item \textbf{Xác định bài toán con:} $P_i$ là bài toán tìm đoạn con bao gồm các phần tử liên tiếp có tổng lớn nhất mà phần tử cuối cùng là $a_i$, với mọi $i = 1, \dots, n$.
\end{itemize}

  \begin{figure}[H] 
    \centering
    \includegraphics[width=0.8\linewidth]{Ex_2.png}
\end{figure} 
\end{frame}
\begin{frame}{Ví dụ 1: Đoạn con có tổng lớn nhất}
\begin{itemize}
    \item \textbf{Bài toán:} Cho dãy $n$ số nguyên $(a_1, a_2, ..., a_n)$. Hãy tìm đoạn con gồm các phần tử liên tiếp sao cho tổng các phần tử là cực đại.
    
    \item \textbf{Xác định bài toán con:} $P_i$ là bài toán tìm đoạn con bao gồm các phần tử liên tiếp của dãy $a_1, a_2, \dots, a_i$ có tổng lớn nhất mà phần tử cuối cùng là $a_i$, với mọi $i = 1, \dots, n$.
    
    \item \textbf{Công thức quy hoạch động} (công thức kết hợp lời giải các bài toán con để thu được lời giải bài toán cha): Gọi $S_i$ là tổng các phần tử của lời giải của $P_i, \forall i = 1, \dots, n$:
    \[
        S_1 = a_1,
    \]
    \[
        S_i = \begin{cases} 
            S_{i-1} + a_i & \text{nếu } S_{i-1} > 0 \\
            a_i & \text{nếu } S_{i-1} \leq 0 
        \end{cases}
    \]
\end{itemize}
\end{frame}

\begin{frame}{Minh họa chạy tay}
  Dãy $A$: $-16, \quad 7, \quad -3, \quad 0, \quad -1, \quad 5, \quad -4$
  
  \begin{itemize}
      \item $S_1 = -16$
      \item $S_2 = a_2 = 7$ (do $S_1 < 0$)
      \item $S_3 = S_2 + a_3 = 7 + (-3) = 4$
      \item $S_4 = S_3 + 0 = 4$
      \item $S_5 = S_4 + (-1) = 3$
      \item $S_6 = S_5 + 5 = 8$
      \item $S_7 = S_6 + (-4) = 4$
  \end{itemize}
  
  $\Rightarrow$ Kết quả: $\max(S) = 8$.
\end{frame}
\begin{frame}{Ví dụ 1: Đoạn con có tổng lớn nhất}
\begin{itemize}
    \item \textbf{Công thức quy hoạch động} (công thức kết hợp lời giải các bài toán con để thu được lời giải bài toán cha):Gọi $S_i$ là tổng các phần tử của lời giải của $P_i, \forall i = 1, ..., n$. Ta có: $S_1 = a_1$,$$S_i = \begin{cases} S_{i-1} + a_i & \text{nếu } S_{i-1} > 0 \\ a_i & \text{nếu } S_{i-1} \le 0 \end{cases}$$
    
    \item \textbf{Lời giải}: Tổng của các phần tử của đoạn con bao gồm các phần tử liên tiếp của dãy có tổng các phần tử được chọn lớn nhất là:\textbf{$$\max(S_1, S_2, ..., S_n) = 8$$}
    
    \item \textbf{Độ phức tạp thuật toán}: $O(n)$
\end{itemize}
\end{frame}
\subsection{Dãy con chung dài nhất}
\begin{frame}{Ví dụ 2: Dãy con chung dài nhất}
    \begin{itemize}
        \item \textbf{Bài toán: }Cho 2 dãy ký tự $X = (x_1, x_2, ..., x_n)$ và $Y = (y_1, ..., y_m)$. Một dãy con của $A$ là một dãy thu được bằng cách xóa đi một số phần tử trong $A$. Hãy tìm độ dài của dãy con chung dài nhất của 2 $X$ và $Y$.
        \item \textbf{Ví dụ:}
        \begin{itemize}
            \item $X = $"$abcb$" và $Y = $"$bdcab$" 
            \item Dãy con chung dài nhất của 2 dãy là dãy "$bcb$" với độ dài 3
        \end{itemize}
        \item \textbf{Nhận xét:} Thuật toán vét cạn, so sánh tất cả các dãy con của $X$ và $Y$ sẽ có độ phức tạp $O(2^n \times 2^m \times \max(m, n))$. Chúng ta có thể giải bài toán này nhanh hơn với một thuật toán quy hoạch động hay không?
    \end{itemize}
\end{frame}
\begin{frame}{Ví dụ 2: Dãy con chung dài nhất}
  \textbf{Bài toán:} Cho 2 dãy $X = (x_1, ..., x_n)$ và $Y = (y_1, ..., y_m)$. Tìm độ dài dãy con chung dài nhất của X và Y (dãy con thu được bằng cách xóa một số phần tử, giữ nguyên thứ tự).
  
  \vspace{0.5em}
  \textbf{Bài toán con:} Gọi $S(i, j)$ là độ dài của dãy con chung dài nhất của 2 dãy, dãy con của X là $X_i = (x_1...x_i)$ với  $i \in \{1, ..., n\}$ và dãy con của $Y$ là $Y_j = (y_1, ..., y_j)$ với $j \in \{1, ..., m\}$.
  
  \vspace{0.5em}
  \textbf{Bài toán cơ sở (Mở rộng chỉ số bắt đầu từ 0 thay vì 1:}
$$S(i, 0) = 0, \forall i \in \{1, ..., n\}$$
$$S(0, j) = 0, \forall j \in \{1, ..., m\}$$
\end{frame}

\begin{frame}{Ví dụ 2: Dãy con chung dài nhất}
  \begin{columns}[c] 
    % Cột chứa text (chiếm 65% chiều rộng)
    \begin{column}{0.5\textwidth}
    \small
  \textbf{Công thức truy hồi:}
  $$ S(i, j) = \begin{cases} 
      S(i-1, j-1) + 1 & \text{nếu } x_i = y_j \\
      \max(S(i-1, j), S(i, j-1)) & \text{ngược lại} 
  \end{cases} $$
    \end{column}
    
    % Cột chứa ảnh (chiếm 33% chiều rộng)
    \begin{column}{0.4\textwidth}
      \centering
      
      \includegraphics[width=\textwidth]{Ex_3.png} 
    \end{column}
  \end{columns}
\end{frame}
\begin{frame}{Ví dụ 2: Dãy con chung dài nhất}
\begin{itemize}
    \item \textbf{Xác định bài toán con:} Gọi $S(i, j)$ là độ dài của dãy con chung dài nhất của 2 dãy, dãy con của $X$ là $X_i = (x_1, ..., x_i)$ với $i \in \{1, ..., n\}$ và dãy con của $Y$ là $Y_j = (y_1, ..., y_j)$ với $j \in \{1, ..., m\}$.
    \item \textbf{Bài toán cơ sở} (mở rộng chỉ số: bắt đầu từ 0 thay vì 1):$$S(i, 0) = 0, \forall i \in \{1, ..., n\}$$$$S(0, j) = 0, \forall j \in \{1, ..., m\}$$
    \item \textbf{Công thức quy hoạch động}:$$S(i, j) = \begin{cases} S(i-1, j-1) + 1 & \text{nếu } x_i = y_j \\ \max\{S(i-1, j), S(i, j-1)\} & \text{ngược lại} \end{cases}$$
    \item \textbf{Độ phức tạp thuật toán}: $O(n \times m)$
\end{itemize}

\end{frame}

\section{Bài tập}
\begin{frame}{Bài tập 1: Dãy con tăng chặt dài nhất}
  \textbf{Đề bài:} Cho dãy số nguyên $A = (a_1, ..., a_n)$ các phần tử đôi một khác nhau. Tìm độ dài của dãy con tăng chặt dài nhất (dãy con $b_1, ..., b_k$ sao cho $b_i < b_{i+1}$).
  
  \vspace{1em}
  \textbf{Hình thức:} 
  \begin{itemize}
      \item Làm tại nhà.
      \item Nộp trên hệ thống chấm code.
  \end{itemize}
\end{frame}

\begin{frame}{Bài tập 2: Dãy con cấp số cộng}
  \textbf{Đề bài:} Cho dãy $A = (a_1, ..., a_n)$. Tìm độ dài dãy con của $A$ là một cấp số cộng với bước nhảy bằng 1 ($d=1$) và có độ dài lớn nhất.
  
  \vspace{1em}
  \textbf{Hình thức:} 
  \begin{itemize}
      \item Làm tại nhà.
      \item Nộp trên hệ thống chấm code.
  \end{itemize}
\end{frame}


%Hết

{\HUSTUseBackground{theme_hust_oneside.pdf}
\begin{frame}
  \ifdefstring{\insertaspectratio}{169}{
    \placecontent{0.355\paperwidth}{0.410\paperheight}{0.640\paperwidth}{
        \color{HUSTRed}\bfseries\fontsize{28pt}{36pt}\selectfont\centering
        THANK YOU!
    }
  }{}
  \ifdefstring{\insertaspectratio}{43}{
    \placecontent{0.355\paperwidth}{0.440\paperheight}{0.640\paperwidth}{
        \color{HUSTRed}\bfseries\fontsize{28pt}{36pt}\selectfont\centering
        THANK YOU!
    }
  }{}
\end{frame}
}

\end{document}